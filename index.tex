\documentclass[preprint,12pt]{elsarticle}

\usepackage[spanish]{babel}
\usepackage{amssymb}
\usepackage{graphicx}
\usepackage{lineno}
\usepackage[utf8]{inputenc}
\usepackage{url}
\usepackage{natbib}

\begin{document}
	
	\begin{frontmatter}

		\title{\huge  COMPARACIÓN DE DOS PARADIGMAS DE PROGRAMACIÓN UTILIZANDO DOS LENGUAJES DE PROGRAMACIÓN}
		\author{Aquino Huallpa, Yaneth Virginia				(20)}
		\author{Chura Velo, Bianka Zugey				(20)}
		\author{Torres Beltran , Johanna Andrea			(2020067849)}
		\author{Esperilla Cruz, Adnner Sleyder				(20)}
		\address{Tacna, Perú}
		
%%INICIO abstract
\begin{abstract}
Nowadays there are languages that concentrate on the ideas of a single paradigm as well as there are others that allow the combination of ideas coming from different paradigms.
\\
In the present article I will propose a comparison between the imperative paradigm using COBOL and the object oriented paradigm using Ruby.
\\
\end{abstract}
%%FIN abstract

\end{frontmatter}

%%INICIO Resumen
\section{Resumen}
En la actualidad existen lenguajes que se concentran en las ideas de un único paradigma así como existen otros que permiten la combinación de ideas provenientes de distintos paradigmas. 
\\
En el presente articulo que planteara una comparación entre el paradigma de imperativo utilizando COBOL y el paradigma orientado a objetos utilizando Ruby.
\\
%%FIN Resumen


%%INICIO Introducción
\section{Introducción}
Los paradigmas de programación permiten determinar la visión y métodos de un programador en la construcción de un programa, por lo que tambien permiten le permiten al programador pensar de dferentes formas en darle solución a el problema, estas soluciones son las que permite la construccion de una aplicación. 
\\
El objetivo de los paradigmas es ayudar a que el programador este mas enfocado en el proceso mental que se realiza para construir un programa que en el programa resultante.
\\
%%FIN Introducción

%%INICIO Marco Teórico
\section{Marco Teórico}
%%----------------------------------------------------------------------------------------------------------------------------------------------------------
	\subsection{\textbf{Paradigma de programación}}
Según Pressman (2010) la calidad de software es el "Proceso eficaz de software que se aplica de manera que crea un producto útil que proporciona valor medible a quienes lo producen y a quienes lo utilizan"(p.340).\cite{referenciatorres1}
\\
\\
Sommerville(2005) citado por López (2015) , escribió se puede definir la calidad del software como “La concordancia con los requerimientos funcionales y de rendimiento explícitamente establecidos, con los estándares de desarrollo explícitamente documentados y con las características implícitas que se espera de todo software desarrollado profesionalmente”(p.46).\cite{referenciatorres2}
\\
\\
Pressman (2005) afirma: "La calidad de software se consigue por medio de la aplicación de métodos de ingeniería de software, prácticas adecuadas de administración y un control de calidad exhaustivo, todo lo cual es apoyado por la infraestructura de aseguramiento de la calidad. En los capítulos que siguen se estudian con cierto detalle el control y aseguramiento de la calidad"(p.352).\cite{referenciatorres1}
\\
\\
Según Pressman (2002) y Sommerville(2006) citado por López (2015), dicen que,"Para lograr la obtencion de un software de buena calidad, es importante que se realice la aplicación de procedimientos estándarizados, para las fases de análisis, diseño, programación y prueba del software. Esto con la finalidad de lograr una mayor confiabilidad, mantenibilidad y facilidad de prueba, y asi aumentar la productividad; tanto para desarrolladores de software como para el control de su calidad"(p.46).\cite{referenciatorres2}
\\

%%****

%%-----------------------------------------------------------------------------
	\subsection{\textbf{ISO/IEC 33000 }}
Es la calidad de los procesos de desarrollo de software que representa a un conjunto de normas internacionales que reemplazan a la norma ISO 15504 - Evaluación y Mejora de la capacidad y madurez de procesos. SPICE ISO/IEC 33000 es una actualización de la serie ISO 15504 DEL AÑO 2003.
\\
El propósito de la serie de estándares ISO/IEC 33000 es proporcionar un enfoque estructurado para la evaluación de procesos, permitiendo a las organizaciones lograr distintos objetivos:
\\
	\begin{itemize}
\item Comprender el estado de sus propios procesos buscando la mejora de los mismos.
\item Determinar la idoneidad de sus propios procesos para un requerimiento en particular o para un conjunto de requerimientos.
\item Determinar la idoneidad de los procesos de otra organización para un contrato específico o para un conjunto de contratos.
	\end{itemize}

\section{Niveles de madurez del ISO/IEC 3300}
En cada nivel existen atributos de evalución a niveles de capacidad:
\\
\begin{itemize}
\item Nivel 0: Proceso incompleto
\item Nivel 1: Basico - Proceso realizado
\\
\section {Divisiones de la norma ISO/IEC 33000}
A continuación se muestra la divisiones que tiene la norma ISO 33000:
\begin{itemize}
\item ISO / IEC 33001 conceptos y terminología para la evaluación de procesos.
	\end{itemize}
	\end{itemize}
%%----------------------------------------------------------------------------------------------------------------------------------------------------------
%%FIN Marco Teórico

	\subsection{\textbf{ISO/IEC 25000}}
	La norma ISO 25000-1, es la más reciente de todas las normativas ISO que afectan al desarrollo del software de calidad. Calidad del software y calidad en el proceso de su desarrollo, deben ir acompañados en cualquier organización que actualmente aspire a unos estándares de calidad óptimos.
\\
La norma ISO 25000 la conforman varias divisiones: 
	\begin{itemize}
		%ITEM1
		\item ISO/IEC 2500n – División de Gestión de Calidad:
		Todas las normas incluidas en este apartado tratarán de definir los modelos, terminología y aspectos que son comunes por todas demás normas de la familia 25000. Esta división se encuentra formada por:
		\begin{itemize}
			\item ISO/IEC 25000 - Guide to SQuaRE: contiene el modelo de la arquitectura de SQuaRE, la terminología de la familia, un resumen de las partes, los usuarios previstos y las partes asociadas, así como los modelos de referencia.
		\end{itemize}
		
		%ITEM 2
		\item ISO/IEC 2501n – División de Modelo de Calidad:
		Este apartado recoge los modelos para la calidad interna, externa y en uso del producto software. Esta división se encuentra formada por:
		\begin{itemize}
			\item ISO/IEC 25010 - System and software quality models: hace la descripción del modelo de calidad para los productos de software y para la calidad en uso.
		\end{itemize}
		
		%ITEM 3
		\item ISO/IEC 2502n – División de Medición de Calidad: Este apartado incluye los modelos de referencia de la medición de la calidad del producto, definiciones de medidas de calidad (interna, externa y en uso) y guías prácticas para su aplicación. La actual división se encuentra formada por:
		\begin{itemize}
			\item ISO/IEC 25020 - Measurement reference model and guide: presenta una explicación introductoria y un modelo de referencia común a los elementos de medición de la calidad. También proporciona una guía para que los usuarios seleccionen o desarrollen y apliquen medidas propuestas por normas ISO.
		\end{itemize}
		
		% ITEM 4
		\item ISO/IEC 2503n – División de Requisitos de Calidad: Las normas que forman este apartado ayudarán a especificar requisitos de calidad que puedan ser utilizados en el proceso de requisitos de calidad del producto software a desarrollar o como entrada del proceso de evaluación. Para ello, este apartado se compone de:
		\begin{itemize}
			\item ISO/IEC 25030 - Quality requirements: provee de un conjunto de recomendaciones para realizar la especificación de los requisitos de calidad del producto software. 

		\end{itemize}
		
		% ITEM 5
		
		\item ISO/IEC 2504n – División de Evaluación de Calidad: Este apartado incluye normas que proporcionan requisitos, recomendaciones y guías para llevar a cabo el proceso de evaluación del producto software. Esta división se encuentra formada por:

		\begin{itemize}
			\item ISO/IEC 25040 - Evaluation reference model and guide: propone un modelo de referencia general para la evaluación, que considera las entradas al proceso de evaluación, las restricciones y los recursos necesarios para obtener las correspondientes salidas.
			\item - ISO/IEC 25041 - Evaluation guide for developers, acquirers and independent evaluators: describe los requisitos y recomendaciones para la implementación práctica de la evaluación del producto software desde el punto de vista de los desarrolladores, de los adquirentes y de los evaluadores independientes.
		\end{itemize}

		


	\end{itemize}

%CONCLUSIONES
\section{Conclusiones}
	\begin{itemize}
\item Utilizar una metodología de aseguramiento de calidad nos ayuda a poder darnos cuenta si efectivamente nuestro software cumple con los requerimientos solicitados, así como validar y controlar el trabajo realizado midiendo atributos como la funcionalidad, capacidad de respuesta frente a errores externos e incluso el nivel de seguridad de la solución.
\item Sin embargo no se puede medir la calidad del software de forma correcta debido a su naturaleza, la certificación se da a los procesos de desarrollo, no al software en sí, el correcto desarrollo de los mismos, garantizaría un buen software. 
	\end{itemize}



	\newpage
	\bibliographystyle{apalike}
	\bibliography{BIBLIOGRAFIA}	


\end{document}
